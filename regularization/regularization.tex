\documentclass[20pt]{article}
\begin{document}
\section{What is Regularization?}

\begin{itemize}
    \item \textbf{Regularization} is a set of techniques used to reduce the complexity of a machine learning model and prevent it from overfitting the training data.
    \item Regularization adds a \textbf{penalty} to the model’s \textbf{loss function}, discouraging it from fitting the training data too perfectly.
    \item Regularization works by adding a \textbf{regularization term} to the loss function (which the model tries to minimize).
\end{itemize}

\subsection{Example}
This is the original loss function (for example):
\begin{equation}
    cost(W) = \frac{1}{2N} \sum_{i=1}^{N} \left( y(X^n, W) - t^n \right)^2
\end{equation}

After adding one of regularization types (ridge for example) it will be:
\begin{equation}
    cost(W) = \frac{1}{2N} \sum_{i=1}^{N} \left( y(X^i, W) - t^i \right)^2 + \mathbf{\sum_{i=1}^{M} \left (\frac{\lambda}{2} W_i^2 \right)}
\end{equation}
The new term is the \textbf{Regularization Term}.

Given one of the weights $W_j$, the partial derivative will be:

\begin{equation}
    \frac{\partial cost(W)}{\partial W_j} = \frac{1}{N} \sum_{i=1}^{N} \left( y(X^i, W) - t^n \right)^2 * X_j^i + \lambda W_j
\end{equation}


\section{Ridge Regression}

\begin{itemize}
    \item We also call it L2 Reguralization.

    \item Adds the squared magnitude of all weights to the loss function:
    \begin{equation}
        \mathbf{Penalty} = \sum_{i=1}^{M} w_j^2
    \end{equation}

    \item It encourages \textbf{smaller weights}.

    \item It keeps all features, but shrinks their influence.
    \item \textbf{Notice} that we started with \textbf{i=1}, which means we don't penalize the intercept at $W_0$.
\end{itemize}

\section{Ridge Regression}

\begin{itemize}
    \item Adds the \textbf{absolute value} of weights to the loss function:
    \begin{equation}
        \mathbf{Penalty} = \sum_{j}^{M} \left| w_j \right|
    \end{equation}

    \item Encourages \textbf{sparsity} — sets some weights to zero.

    \item Can be used for \textbf{feature selection}.

\end{itemize}

\end{document}
